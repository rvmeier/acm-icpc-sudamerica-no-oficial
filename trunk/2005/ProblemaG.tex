\documentclass[a4paper,12pt]{article}

\usepackage{ucs}
\usepackage[utf8x]{inputenc}

\usepackage{url}
\usepackage[spanish]{babel}
\title{Problema G: Computer DJ}
\author{traducido por: Alfredo Paz-Valderrama}
\date{}
\begin{document}
\maketitle

 Un DJ muy famoso, recientemente ha sido invitado a tocar en
la fiesta de gala una conferencia de Ciencia de la
Computacion. Tratando de impresionar a los participantes del
evento, el decidió usar un programa que automáticamente
eligiera las canciones de la fiesta. Sin embargo, el resultado
fue un desastre, debido a la extraña y repetitiva manera en
que el programa elegía las canciones.

 Primero que todo, el DJ seleccionó $N$ canciones del conjunto
de canciones que tenía disponible. Entonces, el programa usado por el
DJ, etiquetó cada cancion usando un caracter
distinto entre ’\verb|A|’ a ’\verb|Z|’. La $i$−ésima canción es etiquetada
usando el $i$−ésimo caracter de la secuencia ’\verb|A|’-’\verb|Z|’. El
programa selecciona las canciones a ser tocadas en la fiesta
en el orden en que sus etiquetas aparecen en la siguiente
secuencia infinita de caracteres: primero todas las palabras
con un caracter en orden lexicográfico; luego todas las
palabras con dos caracteres en orden lexicográfico; luego
todas las palabras con tres caracteres en orden lexicográfico,
y así en adelante. Para $N = 3$, esta secuencia debería ser 
\verb|ABCAAABACBABBBCCACBCCAAAAABAACABAABBABC|\ldots

 Al final de la fiesta, algunas personas preguntaron al DJ, si
él recordaba cual fue la primera canción. Otros quería saber cual fue 
la 25-ava, y así en adelante. El DJ no recordaba nada más
que el extraño patrón de repeticiones, por esto, él requiere de
su ayuda para que ud. diseñe un programa que responda a tales
preguntas.


\section*{Entrada}
 La entrada contiene muchos casos de prueba. Cada caso
consiste de tres líneas. La primera, contiene dos enteros $N$ y
$Q$, indicando el número de canciones elegidas y el número de
consultas hechas por los participantes. ($1 \leq N \leq 26$
y $1 \leq Q \leq 1000$).

En la segunda línea, estarán los $N$ títulos de las canciones
(el título de una canción es una cadena de caracteres alfanuméricos de al menos uno y no más de 100 caracteres) separados por un solo
espacio en blanco. La última línea, contiene una secuencia de
consultas. Cada consulta es un numero $k$
($1 ≤ k ≤ 100000000$) correspondientes a la $k$−ésima
canción tocada en la fiesta. El final de la entrada es indicado
por $N = Q = 0$.
 
\emph{La entrada debería ser leida de la entrada estandar} 

\section*{Salida}
 Para cada consulta $k$ en un caso de prueba, se debería
imprimir una sola línea conteniendo el nombre de la $k$−ésima
canción tocada en la fiesta. Una línea en blanco debería seguir a cada caso de prueba.

\emph{La salida deberá ser escrita a la salida estandar.}\\

\begin{tabular}{|l|l|}
  \hline
  \textbf{Ejemplo de Entrada}&\textbf{Ejemplo de salidad para la entrada}\\
   &\\
 \verb|10 3| &\verb|S2|\\
 \verb|S0 S1 S2 S3 S4 S5 S6 S7 S8 S9| & \verb|S5|\\
 \verb|3 6 10| & \verb|S9|\\
 \verb|3 5|& \\
 \verb|Pathethique TurkishMarch Winter|& \verb|Pathethique|\\
 \verb|1 2 3 4 16|& \verb|TurkishMarch|\\
 \verb|0 0|& \verb|Winter|\\
 & \verb|Pathethique|\\
  \hline
\end{tabular}

\end{document}
